% Options for packages loaded elsewhere
\PassOptionsToPackage{unicode}{hyperref}
\PassOptionsToPackage{hyphens}{url}
\PassOptionsToPackage{dvipsnames,svgnames,x11names}{xcolor}
%
\documentclass[
  a4paper,
]{article}
\usepackage{amsmath,amssymb}
\usepackage{iftex}
\ifPDFTeX
  \usepackage[T1]{fontenc}
  \usepackage[utf8]{inputenc}
  \usepackage{textcomp} % provide euro and other symbols
\else % if luatex or xetex
  \usepackage{unicode-math} % this also loads fontspec
  \defaultfontfeatures{Scale=MatchLowercase}
  \defaultfontfeatures[\rmfamily]{Ligatures=TeX,Scale=1}
\fi
\usepackage{lmodern}
\ifPDFTeX\else
  % xetex/luatex font selection
\fi
% Use upquote if available, for straight quotes in verbatim environments
\IfFileExists{upquote.sty}{\usepackage{upquote}}{}
\IfFileExists{microtype.sty}{% use microtype if available
  \usepackage[]{microtype}
  \UseMicrotypeSet[protrusion]{basicmath} % disable protrusion for tt fonts
}{}
\makeatletter
\@ifundefined{KOMAClassName}{% if non-KOMA class
  \IfFileExists{parskip.sty}{%
    \usepackage{parskip}
  }{% else
    \setlength{\parindent}{0pt}
    \setlength{\parskip}{6pt plus 2pt minus 1pt}}
}{% if KOMA class
  \KOMAoptions{parskip=half}}
\makeatother
\usepackage{xcolor}
\usepackage[margin=1in]{geometry}
\setlength{\emergencystretch}{3em} % prevent overfull lines
\providecommand{\tightlist}{%
  \setlength{\itemsep}{0pt}\setlength{\parskip}{0pt}}
\setcounter{secnumdepth}{5}
\ifLuaTeX
\usepackage[bidi=basic]{babel}
\else
\usepackage[bidi=default]{babel}
\fi
\babelprovide[main,import]{vietnamese}
% get rid of language-specific shorthands (see #6817):
\let\LanguageShortHands\languageshorthands
\def\languageshorthands#1{}
\usepackage{setspace}
\onehalfspacing
\ifLuaTeX
  \usepackage{selnolig}  % disable illegal ligatures
\fi
\usepackage{bookmark}
\IfFileExists{xurl.sty}{\usepackage{xurl}}{} % add URL line breaks if available
\urlstyle{same}
\hypersetup{
  pdftitle={Đồ án -- Phân lớp},
  pdflang={vi},
  colorlinks=true,
  linkcolor={Maroon},
  filecolor={Maroon},
  citecolor={Blue},
  urlcolor={red},
  pdfcreator={LaTeX via pandoc}}

\title{Đồ án -- Phân lớp}
\usepackage{etoolbox}
\makeatletter
\providecommand{\subtitle}[1]{% add subtitle to \maketitle
  \apptocmd{\@title}{\par {\large #1 \par}}{}{}
}
\makeatother
\subtitle{Multivariate Statistical Applied}
\author{}
\date{\today}

\begin{document}
\maketitle
\begin{abstract}
This is an abstract!
\end{abstract}

\pagebreak
\tableofcontents 
\pagebreak

\section{Giới thiệu}\label{giux1edbi-thiux1ec7u}

Trong các bài toán phân tích và tái tổ chức dữ liệu, hay các bài toán
phân loại (classification issues), liên quan đến việc nhóm và phân loại
các đối tượng (hoặc dữ liệu) dựa trên các đặc trưng (features) của
chúng, có thể tiếp cận theo hai góc nhìn.

Thứ nhất, bài toán phân cụm (Clustering) là quá trình phân chia các đối
tượng thành các nhóm tự nhiên mà không biết trước nhóm hay nhãn. Việc
phân chia này được thực hiện dựa trên việc phân tích cấu trúc của tập dữ
liệu và nhóm các đối tượng có sự tương đồng cao vào cùng một nhóm.

Khía cạnh thứ hai là bài toán phân loại (Classification), trong đó có
hai quá trình chính: nghiên cứu và xây dựng một hàm phân loại để phân
biệt các đối tượng, và gán nhãn cho các nhóm, với nhãn đã biết trước.

Trong bài nghiên cứu này, chúng tôi sẽ tập trung vào bài toán phân lớp
theo góc nhìn thứ hai, cụ thể là nghiên cứu phương pháp phân tích phân
biệt (Discriminant Analysis) và quá trình phân loại các đối tượng thành
các nhóm với nhãn đã biết trước.

\subsection{Phân tích phân biệt (Discriminant Rules) và Phân tích phân
loại (Classification
Rules)}\label{phuxe2n-tuxedch-phuxe2n-biux1ec7t-discriminant-rules-vuxe0-phuxe2n-tuxedch-phuxe2n-loux1ea1i-classification-rules}

\subsection{Phát biểu bài toán}\label{phuxe1t-biux1ec3u-buxe0i-touxe1n}

Giả sử có \(J\) quần thể (populations), ký hiệu \(j = 1, 2, \dots, J\),
cần gán một quan sát \(x\) vào một trong các nhóm này.

\begin{itemize}
\item
  \textbf{Mục tiêu}: Xây dựng một hàm phân loại\\
  \(f: \mathbb{R}^d \to \{1, 2, \dots, J\}\) để dự đoán nhãn \(y\) của
  một quan sát mới \(x^* \in \mathbb{R}^d\), gồm hai bước chính:

  \begin{enumerate}
  \def\labelenumi{\arabic{enumi}.}
  \tightlist
  \item
    \textbf{Phân biệt các lớp (Discrimination):} Xác định đặc trưng giúp
    phân biệt các nhóm, từ đó thiết lập quy tắc phân biệt (discriminant
    rules).\\
  \item
    \textbf{Phân loại dữ liệu mới (Classification):} Gán nhãn \(y\) cho
    quan sát mới dựa trên các quy tắc đã học.\\
  \end{enumerate}
\item
  \textbf{Mục đích}: Tách biệt các nhóm dữ liệu và phân bổ chính xác các
  quan sát mới vào các nhóm đã biết, đồng thời tối ưu hóa mô hình để
  giảm sai số phân loại hoặc cực đại hóa xác suất hậu nghiệm \(P(y|x)\).
\item
  \textbf{Đầu vào}:

  \begin{itemize}
  \tightlist
  \item
    Một tập dữ liệu huấn luyện gồm \(n\) quan sát:
    \[\{(x_i, y_i)\}_{i=1}^{n}\] trong đó:

    \begin{itemize}
    \tightlist
    \item
      \(x_i \in \mathbb{R}^d\) là một điểm dữ liệu có \(d\) đặc trưng
      (features).
    \item
      \(y_i \in \{1, 2, \dots, J\}\) là nhãn (label) của \(x_i\), thuộc
      một trong \(J\) nhóm.
    \end{itemize}
  \item
    Một quan sát mới \(x^* \in \mathbb{R}^d\) cần được phân loại.
  \end{itemize}
\item
  \textbf{Đầu ra}:

  \begin{itemize}
  \tightlist
  \item
    Một mô hình phân loại \(f: \mathbb{R}^d \to \{1, 2, \dots, J\}\)
  \item
    Nhãn dự đoán \(y^* = f(x^*)\) cho quan sát mới \(x^*\).
  \end{itemize}
\item
  \textbf{Lưu ý}: Mặc dù lý thuyết phân biệt và phân loại có sự khác
  nhau, trong thực tế, chúng thường kết hợp và hỗ trợ lẫn nhau: mô hình
  giúp phân biệt nhóm dữ liệu cũng có thể được dùng để phân loại, và
  ngược lại, một mô hình phân loại tốt thường phản ánh rõ các yếu tố
  phân biệt giữa các nhóm.
\end{itemize}

\section{Allocation Rules for Known
Distributions}\label{allocation-rules-for-known-distributions}

\section{Discrimination Rules in
Practice}\label{discrimination-rules-in-practice}

\end{document}
